\documentclass[a4paper,12pt]{extarticle}

\pdfmapfile{pscyr.map}
\renewcommand{\rmdefault}{ftm}


\usepackage[titletoc]{appendix}
\usepackage{indentfirst}
\usepackage[T2A]{fontenc}
\usepackage[utf8]{inputenc}
\usepackage[russian]{babel}
\usepackage{graphicx}
\usepackage{titlesec}
\usepackage{enumerate}
\usepackage{enumitem}
\usepackage{cleveref}
\usepackage[tableposition=top]{caption}
\usepackage{subcaption}
\usepackage{chngcntr}
\usepackage{float}
\usepackage{textcase} 
\usepackage{tocloft}
\usepackage{listings}
\usepackage{lastpage}
\usepackage{soulutf8}
\usepackage[figure,table,section]{totalcount}
\usepackage[section]{placeins}
\usepackage{pdfpages}
\usepackage{cleveref}
\usepackage{algorithm}
\usepackage{tabularx}
\usepackage{multirow}
\usepackage{makecell}
\usepackage{tikz}
\usepackage{pgfplots}
\usepackage{courier}
\usepackage{microtype}
\usepackage[scaled]{beramono}
\usepackage{ifthen}
\usepackage{varwidth}
\usetikzlibrary{shapes,arrows,matrix}
\usepackage{geometry} % Меняем поля страницы
\geometry{left=2.5cm}% левое поле
\geometry{right=1.5cm}% правое поле
\geometry{top=2cm}% верхнее поле
\geometry{bottom=2cm}% нижнее поле

\newcommand\myappendix[1]{

\refstepcounter{section}

\section*{Приложение~\thesection{}~#1}

\addcontentsline{toc}{section}{~\thesection{}~#1}}

% Для списков
% \renewcommand{\labelenumi}{-}
% \renewcommand{\labelenumii}{\arabic{enumii})}
% \renewcommand{\labelenumiii}{\arabic{enumii}.\arabic{enumiii})}

\setlength\parindent{1.27cm}

\makeatletter
\AddEnumerateCounter{\asbuk}{\russian@alph}{щ}
\makeatother

\newlength\myfntht
\newcommand{\myfontsize}[2][1.2]{\setlength\myfntht{#2}%
	\fontsize{\myfntht}{#1\myfntht}\selectfont}

% Отступы для списков.
%\setenumerate[1]{label=\asbuk*), ref=\asbuk*, fullwidth, itemindent=\parindent, leftmargin=0\parindent, topsep=0pt}

\setenumerate[1]{label=-, fullwidth, itemindent=\parindent, leftmargin=0\parindent, topsep=0pt, itemsep=-5pt}

\setenumerate[2]{label=\arabic{enumii}), fullwidth, itemindent=1\parindent, leftmargin=1\parindent}

\setenumerate[3]{label=\arabic{enumii}.\arabic{enumiii}), fullwidth, itemindent=1\parindent, leftmargin=1\parindent}

% Межстрочный интервал
\renewcommand{\baselinestretch}{1.25}

\sloppy

%Частота переносов
\hyphenpenalty=2000

%Висячие строки
\clubpenalty=10000
\widowpenalty=10000

% Нумерация страниц сверху
\pagestyle{myheadings}

%Точки в содержании
\renewcommand{\cftsecleader}{\cftdotfill{\cftdotsep}}

\linespread{1.3}
\frenchspacing

\newlength{\normalparindent}
\AtBeginDocument{\setlength{\normalparindent}{\parindent}}

\titlespacing*{\chapter}
  {0pt}{0ex}{1\baselineskip}
\titlespacing*{\section}
  {0pt}{0ex}{1\baselineskip}
\titlespacing*{\subsection}
  {0pt}{2\baselineskip}{2\baselineskip}
  \titlespacing*{\subsubsection}
  {0pt}{2\baselineskip}{2\baselineskip}

% 
\titleformat{\chapter}[block]
  {\centering}
  {\hspace*{\normalparindent}\thechapter}
  {1ex}
  {}

\titleformat{\section}[block]
  {\centering}
  {\hspace*{\normalparindent}\thesection}
  {1ex}
  {}

\titleformat{\subsection}[block]
  {}
  {\hspace*{\normalparindent}\thesubsection}
  {1ex}
  {}
  
\titleformat{\subsubsection}[block]
  {}
  {\hspace*{\normalparindent}\thesubsubsection}
  {1ex}
  {}
 

% 
%\allsectionsfont{\normalsize}

 
\DeclareCaptionLabelFormat{gostfigure}{Рисунок #2}
\DeclareCaptionLabelFormat{gosttable}{Таблица #2}
\DeclareCaptionLabelSeparator{gost}{~---~}
\captionsetup{labelsep=gost}
\captionsetup[figure]{labelformat=gostfigure}
\captionsetup[table]{labelformat=gosttable}
\renewcommand{\thesubfigure}{\asbuk{subfigure}}

\newcommand{\crefrangeconjunction}{-}
\crefname{table}{см. рис.}{см. рис.}

%For counting
\counterwithin{figure}{section}
\counterwithin{table}{section}

\graphicspath{ {img/} }

\newcommand{\addimg}[4]{
\begin{figure}[!htb]
    \centering
    \captionsetup{justification=centering}
    \includegraphics[scale=#2]{#1}
    \caption{#3}
    \label{fig:#4}
\end{figure}
}

\newcommand{\addtable}[3]{
	\begin{table}[!htb]
		\centering
		\captionsetup{justification=centering}
		\caption{#1}
		#3	
		\label{table:#2}
	\end{table}
}

\crefformat{figure}{см. рис. #1}
\crefformat{table}{см. табл. #1}

\newcommand\Small{\fontsize{9}{9.2}\selectfont}
\newcommand*\LSTfont{\Small\ttfamily\SetTracking{encoding=*}{-60}\lsstyle}

\lstset { 
	language=C++,
	basicstyle=\LSTfont,
	tabsize=2
}

%%%%% Tikz settings for grids

\tikzset{>=latex}

\pgfkeys{
	mygrid/.is family,
	mygrid,
	width/.initial=2,
	height/.initial=2,
	step/.initial=1,
	color/.initial=black,
}
\newcommand\mygridset[1]{\pgfkeys{mygrid,#1}}
\newcommand\mygrid[1][]{
	\mygridset{#1,
		width/.get=\gridwidth,
		height/.get=\gridheight,
		step/.get=\gridstep,
		color/.get=\gridcolor
	}
	
	\draw [step=\gridstep, thick,\gridcolor]
	(0,0) grid (\gridwidth,\gridheight);
}

\pgfkeys{
	mybox/.is family,
	mybox,
	x/.initial=0.5,
	y/.initial=0.5,
	color/.initial=blue,
}

\newcommand\myboxset[1]{\pgfkeys{mybox,#1}}
\newcommand\mybox[1][]{
	\myboxset{#1,
		x/.get=\boxx,
		y/.get=\boxy,
		color/.get=\boxcolor
	}
	\node[rectangle, minimum size=1cm, fill=\boxcolor, draw=none] at (\boxx + 0.5, \boxy + 0.5){};
}

\pgfkeys{
	myarrow/.is family,
	myarrow,
	startx/.initial=0.5,
	starty/.initial=0.5,
	endx/.initial=1.5,
	endy/.initial=1.5,
	color/.initial=red,
}

\newcommand\myarrowset[1]{\pgfkeys{myarrow,#1}}
\newcommand\myarrow[1][]{
	\myarrowset{#1,
		startx/.get=\arrowstartx,
		starty/.get=\arrowstarty,
		endx/.get=\arrowendx,
		endy/.get=\arrowendy,
		color/.get=\arrowcolor
	}
		
	\draw[line width=0.1cm,->,shorten <=0.2cm, \arrowcolor] (\arrowstartx + 0.5, \arrowstarty + 0.5) -- (\arrowendx + 0.5, \arrowendy + 0.5);	
}

\definecolor{grey}{RGB}{200,200,200}
\definecolor{darkgrey}{RGB}{110,110,110}

\newcommand{\addtikz}[4]{
	\begin{figure}[!htb]
		\centering
		\captionsetup{justification=centering}
		
		\begin{tikzpicture}[scale=#3, every node/.style={scale=#3}]
		#4
		\end{tikzpicture}
		
		\caption{#1}
		\label{fig:#2}
	\end{figure}
}

%%%%% End tikz settings 

\makeatletter
\renewenvironment{thebibliography}[1]
{\list{\@arabic\c@enumiv.}%
    {\settowidth\labelwidth{\@biblabel{#1}}%
    	\topsep=0pt
        \leftmargin=0pt
        \itemindent=1.5\parindent	% Костыль
        \itemsep=-5pt	% Костыль
        \@openbib@code
        \usecounter{enumiv}%
        \let\p@enumiv\@empty
        \renewcommand\theenumiv{\@arabic\c@enumiv}}%
    \sloppy
    \clubpenalty4000
    \@clubpenalty \clubpenalty
    \widowpenalty4000%
    \sfcode`\.\@m}
{\def\@noitemerr
    {\@latex@warning{Empty `thebibliography' environment}}%
    \endlist}
\makeatother

\newcounter{mycite}
\newtoks\citetoks
\makeatletter
\DeclareRobustCommand\unscite[1]{%
	\@ifundefined{uns@cite#1}
	{\refstepcounter{mycite}\label{citelabel@#1}%
		\expandafter\xdef\csname uns@cite#1\endcsname{\arabic{mycite}}%
		\toks\z@=\expandafter{\the\citetoks}%
		\toks\tw@=\expandafter\expandafter\expandafter{%
			\csname uns@bibitem#1\endcsname}%
		\edef\@tempcite{\the\toks\z@\the\toks\tw@}%
		\global\citetoks=\expandafter{\@tempcite}%
	}{}[\@nameuse{uns@cite#1}]}
\newcommand{\orderedbibitem}[2]{%
	\@namedef{uns@bibitem#1}{\bibitem{#1}#2}}
\makeatother

\renewcommand{\cite}{\unscite}

\orderedbibitem{HPA}{Botea A. Near Optimal Hierarchical Path-Finding: статья/ Botea A., Muller M., Schaeffer J. - University of Alberta - 30с.}

\orderedbibitem{IMPROVING_JPS}{
	Harabor D. Improving Jump Point Search: статья/ Harabor D., Grastien A. - The Australian National University - 8с.}

\orderedbibitem{HEURISTICS}{
   	Информационный портал stanford.edu - Heuristics From Amit’s Thoughts on Pathfinding [Электронный ресурс]: -- Режим доступа: https://theory.stanford.edu/~amitp/GameProgramming/Heuristics.html - Загл. с экрана.}

\orderedbibitem{GOAL_BOUNDING}{
   	Репозиторий кода GitHub - Steve Rabin JPS+ and Goal Bounding [Электронный ресурс]: -- Режим доступа: https://www.github.com/SteveRabin/JPSPlusWithGoalBounding/ - Загл. с экрана.}

\orderedbibitem{UML_USER_GUIDE_2ND}{
   	Unified Modeling Language User Guide, The, 2nd Edition
   	/ G. Booch, J. Rumbaugh, I. Jacobson - Addison-Wesley Professional, 2005. - 496 с. - ISBN 0-321-26797-4.}

\begin{document}

\thispagestyle{empty}

{
\centering{
	\textbf{
		\myfontsize{14pt}{МІНІСТЕРСТВО ОСВІТИ І НАУКИ УКРАЇНИ	}}


	\textbf{
		\myfontsize{13pt}{ХАРКІВСЬКИЙ НАЦІОНАЛЬНИЙ УНІВЕРСИТЕТ РАДІОЕЛЕКТРОНІКИ }}
	
	\vspace{1\baselineskip}
	
	\myfontsize{14pt}{
	
		Факультет комп’ютерних наук
		
		\vspace{2\baselineskip}
		
		Кафедра Програмної інженерії
		
		\vspace{2\baselineskip}
		
		\textbf{ВИПУСКНА КВАЛІФІКАЦІЙНА РОБОТА БАКАЛАВРА}
		
		\vspace{4\baselineskip}}
	
\myfontsize{16pt}{ Програма пошуку найкоротшого шляху для iгрових додаткiв }


\myfontsize{8pt}{(Тема  роботи)}

\vspace{1\baselineskip}

\myfontsize{14pt}{\quad\quad Студент гр.  ПІ-12-1 \hfill Піляєв Д.В. \quad \quad}

\vspace{-0.5\baselineskip}

\flushleft{
\myfontsize{10pt}{\hspace{10em} позначка групи \hspace{21em}  Прізвище, ініціали }	}

\vspace{1\baselineskip}

\myfontsize{14pt}{\quad\quad Керівник роботи, проф. \hfill Качко О.Г. \hspace{1em} \quad}

\vspace{-0.5\baselineskip}

\flushleft{
	\myfontsize{10pt}{\hspace{14em} звання \hspace{21em}  Прізвище, ініціали }	}


\vspace{4\baselineskip}

\flushleft{\myfontsize{14pt}{\quad\quad\textbf{Допускається до захисту}}}

\flushleft{
	\myfontsize{14pt}{\quad\quad Зав. кафедри, проф. \quad\quad \quad  \rule[-0.7ex]{8em}{.3pt} \hfil \underline{\hspace{2em} З. В.Дудар \hspace{2em}} }
}

\vfill
\centering{
2016 р.}

}
}

\newpage

\thispagestyle{empty}

{
	
\myfontsize{12pt}{
\centering{	\textbf{
	ХАРКІВСЬКИЙ НАЦІОНАЛЬНИЙ УНІВЕРСИТЕТ РАДІОЕЛЕКТРОНІКИ }}	
	
\vspace{1\baselineskip}

\underline{Факультет комп’ютерних наук} \hfill \underline{Кафедра програмної інженерії} \quad
	
\flushleft	
\textbf{Напрям} \underline{6.050103 «Програмна інженерія»}

\vspace{1\baselineskip}

\hfill\begin{minipage}{0.4\linewidth}
	\centering
	
	ЗАТВЕРДЖУЮ:
	
	\ul{\mbox{\hspace{9em}}}
	
	"\ul{\mbox{\hspace{2em}}}" \ul{\mbox{\hspace{4em}}}20\ul{\mbox{\hspace{1.5em}}}р
	
	Зав. кафедри проф. З.В.Дудар
\end{minipage}

\vspace{2\baselineskip}

\centering{ 	\textbf{ЗАВДАННЯ}
		
	\textbf{НА ВИПУСКНУ КВАЛІФІКАЦІЙНУ РОБОТУ БАКАЛАВРА СТУДЕНТОВІ }
}

\centering{\underline{Піляєв Данило Вікторович}}
%\underline{\hspace{\linewidth}}

\centering{\myfontsize{8pt}{Прізвище, ім’я, по батькові} }

\vspace{1\baselineskip}

\begin{enumerate}
	[
	labelindent=*,
	style=multiline,
	leftmargin=*,
	label=\arabic*.
	]
	
	\item Тема роботи «Програма пошуку найкоротшого шляху для ігрових додатків» затверджена наказом університету № \underline{\quad NNN \quad} від «\ul{15}» \ul{травня} 2016 р.
	
	\item Термін здачі студентом закінченої роботи «\ul{15}» \ul{червня} 2016 р.
	
	\item Вихідні дані до роботи: \ul{розробити програму пошуку найкоротшого шляху для ігрових додатків, використовуючи мову C++}
	
	\item Зміст пояснювальної записки:
	
	\item Перелік графічного матеріалу (назви слайдів презентації)
\end{enumerate}

}
}

\newpage
\thispagestyle{empty}

\begin{enumerate}
	[
	labelindent=*,
	style=multiline,
	leftmargin=*,
	label=\arabic*.
	]
	\setcounter{enumi}{5}
	
	\item Консультанти з (роботи) із зазначенням розділів проекту, що їх стосуються
	
	\begin{table}[!h]
	\begin{tabularx}{\textwidth}{XXXXl}
		\cline{1-4}
		\multicolumn{1}{|l|}{\multirow{2}{*}{Розділ}} & \multicolumn{1}{l|}{\multirow{2}{*}{Консультант}} & \multicolumn{2}{l|}{Підпис, дата}                                           &  \\ \cline{3-4}
		\multicolumn{1}{|l|}{}                        & \multicolumn{1}{l|}{}                             & \multicolumn{1}{l|}{Завдання видав} & \multicolumn{1}{l|}{Завдання прийняв} &  \\ \cline{1-4}
		\multicolumn{1}{|l|}{Спецчастина}             & \multicolumn{1}{l|}{}                             & \multicolumn{1}{l|}{Качко О.Г.}     & \multicolumn{1}{l|}{}                 &  \\ \cline{1-4}
		&                                                   &                                     &                                       & 
	\end{tabularx}
	\end{table}
	
	\item Календарний план
	
	\begin{table}[!h]
	\begin{tabularx}{0.975\textwidth}{|l|p{6.5cm}|p{4cm}|X|}
		\hline
		№ & Назва етапів бакалаврської роботи & Термін виконання & \myfontsize{10pt}{\thead{Позначка керівника\\ про реальний строк\\ виконання}} \\ \hline
		1 & & «20» квітня 2016р \newline \myfontsize{8pt}{День початку практики}   &        \\ \hline
		2 & & «20» квітня 2016р  &        \\ \hline
		3 & & «20» квітня 2016р  &        \\ \hline
		4 & & «20» квітня 2016р  &        \\ \hline
		5 & & «20» квітня 2016р  &        \\ \hline
		6 & & «20» квітня 2016р  &        \\ \hline
		7 & Підготовка пояснювальної записки. & «20» квітня 2016р  &        \\ \hline
		8 & Спецчастина & «20» квітня 2016р &    \\ \hline
		9 & Підготовка презентації та доповіді  & «20» квітня 2016р & \\ \hline
		10 & Рецензування  & «20» квітня 2016р \newline \myfontsize{8pt}{за два тижня до початку захистів} & \\ \hline
		11 & Попередній захист   & «20» квітня 2016р \newline \myfontsize{8pt}{за тиждень до початку захистів} & \\ \hline
		12 & Занесення диплома в електронний архів   & «20» квітня 2016р \newline \myfontsize{8pt}{за 5 днів до початку роботи ДЕК} & \\ \hline
		13 & Допуск до захисту у зав. кафедри & «20» квітня 2016р \newline \myfontsize{8pt}{за 3 дні до початку роботи ДЕК} & \\ \hline
	\end{tabularx}
	\end{table}
\end{enumerate}

Дата видачі завдання «\underline{20}» квітня 2016 р.

Керівник \ul{\mbox{\hspace{5em}}} \hspace{7em} \ul{\mbox{\hspace{6em}}} \hspace{1em} \ul{\mbox{\hspace{8em}}}

\hspace{4em} {\myfontsize{8pt}{посада, звання}} \hspace{17em} {\myfontsize{8pt}{Прізвище, ініціал}}

\vspace{1\baselineskip}

Завдання прийняв до виконання \hspace{2.5em} \ul{\mbox{\hspace{6em}}} \hspace{1em} \ul{\mbox{\hspace{8em}}}

\hspace{26em} {\myfontsize{8pt}{Прізвище, ініціал}}

\newpage

%\includepdf[pages=1]{title.pdf}
%
%\includepdf[pages=1-2]{task.pdf}

\section*{РЕФЕРАТ}

\vspace{1\baselineskip}

\thispagestyle{empty}

Звіт з переддипломної практики бакалаврів: \pageref{LastPage} сторінок, \totalsections\ розділів,  \totalfigures\ рисунків, 10 джерел.

Об'єкт проектування -- програма для пошуку найкоротших шляхів.

Метою роботи є створення оптимізованної програми для пошуку найкоротших шляхів в ігрових додатків.

Методи розробки базуються на язиці С++11, та бібліотеках threadpool11, SFML і SFGUI

У результаті роботи була здійснена програмна реалізація програми для пошуку шляхів в ігрових додатків, а саме алгоритмів A*, JPS та Goal Bounding. В процессі ці алгоритми були проаналізовані та оптимізовані.

БИБЛИОТЕКА, ПОИСК ПУТЕЙ, C++, A*, JPS, GOAL BOUNDING, HAA*, КВАДРАТНАЯ СЕТКА, ЭВРИСТИЧЕСКАЯ ФУНКЦИЯ.

%\vspace{1\baselineskip}
%
%Explanatory note: \pageref{LastPage} pages, \totalsections\ sections,  \totalfigures\ figures, 8 sources, 1 application.
%
%Subject of architecturing -- library for path finding.
%
%The aim is to create optimized library for path finding in games.
%
%The development methods are based on С++11 and libraries threadpool11, SFML and SFGUI.
%
%The result of work is implemented library for path finding in games which includes A* and JPS algorithms. During development this algorithms were analyzed and optimized.
%
%LIBRARY, ПPATH FINDING, C++, A*, JPS, GOAL BOUNDING, HAA*, SQUARE GRID, HEURISTIC FUNCTION.

\clearpage

\section*{СОДЕРЖАНИЕ}

%\thispagestyle{empty}

% Ужасные костыли...

\setlength{\cftbeforesecskip}{0pt}
\renewcommand{\cftsecpagefont}{\normalfont}

\renewcommand{\cfttoctitlefont}{\hspace{0.38\textwidth} \MakeUppercase}
\renewcommand{\cftbeforetoctitleskip}{-1em}
\renewcommand{\cftsecfont}{\hspace{2.5em}}
\renewcommand{\cftsubsecfont}{\hspace{1em}}

\renewcommand{\cftparskip}{-1mm}
\renewcommand{\cftdotsep}{1}
\setcounter{tocdepth}{2} 
\renewcommand*\contentsname{}
 
\cftsetindents{section}{-0.8\parindent}{1em}
\cftsetindents{subsection}{-0.3\parindent}{2em}
\cftsetindents{subsubsection}{0.05\parindent}{3em}

\thispagestyle{empty}
\tableofcontents

\thispagestyle{empty}
    
\clearpage

\section*{ВВЕДЕНИЕ}
\addcontentsline{toc}{section}{Введение}

\vspace{1\baselineskip} 

Задачей нахождения кратчайшего пути является поиск оптимального и короткого пути между двумя точками. Проблема нахождения кратчайших путей возникает в таких случаях как: оптимизация перевозки грузов и пассажиров, оптимальная маршрутиризация пакетов в сети, навигация искусственного интеллекта и игрока в компьютерных играх, а так же навигация роботов в пространстве. Все компьютерные игры имеют поиск путей в том или ином виде, поэтому скорость и точность алгоритма часто влияет на качество искусственного интеллекта и восприятия игрока. 

Существует несколько представлений пространства для проведения поиска путей, одним из них является квадратная сетка, которая проста для создания и используется в стратегиях реального времени и других играх с двумерной картой. Хотя квадратная сетка в большинстве случаев является неоптимальным представлением области поиск, с ней очень просто работать и легко модифицировать, что значительно упрощает программную работу с игровой картой. В следствии неоптимальности представления карты появляется необходимость в оптимизации алгоритмов поиска работающих с ней посредством общей оптимизации логики работы алгоритма, введение некоторых допущений и ограничений на область поиска, а так же проведение низкоуровневых оптимизаций.

Таким образом, задача нахождения кратчайшего пути на области представленной квадратной сеткой является актуальной проблемой, которая будет рассмотрена и исследована в данной работе.

Целью преддипломной практики является проведение анализа существующих алгоритмов поиска путей, разработка различных вариантов алгоритмов и их оптимизаций, создание оптимизированной универсальной библиотеки для нахождения оптимальных маршрутов и анализ полученных результатов.



\clearpage

%\section[Организационная структура предприятия]{ОРГАНИЗАЦИОННАЯ СТРУКТУРА ПРЕДПРИЯТИЯ}

\vspace{1\baselineskip} 
Программа пошуку нойкоротшого шляху для игрових додаткив
Харьковский национальный университет радиоэлектроники — ХНУРЭ.
В университете обучается более 12 тысяч студентов по 34 специальностям. ХНУРЭ включает 9 факультетов, и 30 кафедр. 

Преддипломная практика проходила на базе кафедры программной инженерии (ПИ). В состав кафедры входят четыре научные лаборатории.

 Кафедра программной инженерии является профилирующей при подготовке бакалавров по направлению «Программная инженерия» (ПИ), магистров и специалистов по специальности «Инженерия программного обеспечения» (ИПО), «Программное обеспечение систем» (ПОС). Стандарт образования по направлению «программная инженерия» полностью отвечает европейскому стандарту «Software Engineering».

Кафедра программной инженерии сотрудничает в сфере обучения и науки в IT с Институтом математики и системной инженерии (School of Mathematics and Systems Engineering) в Швеции.

На базе научных лабораторий, лаборатории Бизнес-Инкубатор и учебно-научного отряда "Программист" организовываются семинары и кружки в рамках которых студенты имеют возможность повысить свой профессиональный уровень.
%
%\clearpage

\section{\MakeTextUppercase{АНАЛИЗ ПРЕДМЕТНОЙ ОБЛАСТИ}}
\subsection{Алгоритмы нахождения кратчайших путей}

Задачей нахождения кратчайшего пути является поиск оптимального и короткого пути между двумя точками. Решения этой задачи в большинстве случаев основаны на алгоритме Дейкстры для нахождения кратчайшего пути в взвешенных графах. 

Простейшие алгоритмы для обхода графа, такие как поиск в ширину и поиск в глубину могут найти какой-то путь от начальной до конечной вершины, но не учитывают стоимость пути. Одним из первых алгоритмов поиска пути с учётом его стоимости был алгоритм Беллмана -- Форда, который проходит по всем возможным маршрутам и находит наиболее оптимальный, вследствие чего имеет временную сложность $O(|V||E|)$, где $V$ - количество вершин, а $E$ - количество рёбер. Однако для нахождения пути близкого к оптимальному не обязательно перебирать все пути, а можно отсекать неперспективные направления на основе некой эвристики, что может дать таким алгоритмам нижнюю оценку $O(|E|\log(|V|))$. Такими алгоритмами являются алгоритм Дейкстры, A* и их модификации.

\subsubsection{A*}

Алгоритм A{*} (A звёздочка) - это алгоритм общего назначения, который может быть использован для решения многих задач, например для нахождения путей. A{*} является вариацией алгоритма Дейкстры и используя эвристическую функция для ускорения работы, при этом гарантируя наиболее эффектное использование онной. 

Алгоритм A{*} поочерёдно рассматривает наиболее перспективные неисследованные точки или точки с неоптимальным маршрутом до них, выбирая пути которые минимизируют $ f(n) = g(n) + h(n) $, где $n$ последняя точка в пути, $g(n)$ - стоимость пути от начальной точки до точки $n$, а $h(n)$ - эвристическая оценка стоимости пути от $n$ до конца пути. Когда точка исследована, алгоритм останавливается если это конечная точка, иначе все её соседи добавляются в список для дальнейшего исследования.

Для нахождения пути от начальной до конечной точки, кроме стоимости пути до точки, следует записывать и её предка (точку из которой мы пришли в неё).

%\begin{algorithm}
%    \caption{Псевдокод A*}\label{alg:a-star-example}
%    \begin{algorithmic}[1]
%        \State Добавляем начальную точку в открытый список.
%        \Repeat
%        \State Поиск точки из открытого списка с наименьшим значением функции f.
%        \State Добавить точку в закрытый список.
%        \ForAll{точек соседних текущей}
%        \If{точка входит в закрытый список}
%        \State Игнорировать её.
%        \ElsIf{точка не в открытом списке}
%        \State Добавляем точку в открытый список. 
%        \State Делаем текущую точку предком этой точки. 
%        \State Подсчитываем её параметры.
%        \Else
%        \If{возможно ли улучшить длину пути через неё} 
%        \State Меняем её предка и пересчитываем параметры.
%        \EndIf
%        \EndIf
%        \EndFor
%        \Until Конечная точка не добавлена в закрытый список и открытый список не пуст.
%    \end{algorithmic}
%\end{algorithm}

Свойства алгоритма A*:

\begin{enumerate}
    \item Алгоритм гарантирует нахождение пути между точками, если он существует;
    \item Если эвристическая функция $h(n)$ не переоценивает действительную минимальную стоимость пути, то алгоритм работает наиболее оптимально;
    \item A* оптимально эффективен для заданной эвристики $h(n)$.
\end{enumerate}

На оценку сложности A* влияет использованная эвристика, в худшем случае количество точек рассмотренных A* экспоненциально растёт по сравнению с длинной пути, однако при эвристике $h(n)$ удовлетворяющей условию $|h(x) - h^*(x)| \leq O(\log h^*(x))$, где $h^*(n)$ - оптимальная эвристика, алгоритм будет иметь полиномиальную сложность. Так же на временную сложность влияет выбранный способ хранения закрытых и открытых точек.

\underline{Оценка памяти}

\subsubsection{JPS}



\subsubsection{HPA* и HAA*}

HPA* (Hierarchical Path-Finding A*) - добавляет алгоритму A* иерархическую абстракцию, разбивая карту на прилегающие друг к другу кластеры, которые соединены входами. Одной из основных идей алгоритма является то, что расчёт пути в A* каждый раз происходит с нуля, что можно исправить добавив сохранение кратчайших путей между определёнными точками. 

Первым этапом алгоритма является препроцессинг карты для построения кластеров и их входов. При этом возможно построение нескольких уровней графа кластеров используя один и тот же алгоритм рекурсивно на созданных во время предыдущего прохода кластерах.

Во время исполнения программы запрос нахождения пути выполняется рекурсивно находя и уточняя путь на графе начиная с самых крупного уровня кластеров. После нахождения пути может применятся его сглаживание.

Алгоритм HPA* работает с такими допущениями:

\begin{enumerate}
    \item Все актёры имеют одинаковый размер, при этом все части навигационной сетки проходимы ими;
    \item На всех участках карты агенты имеют одинаковую проходимость.
\end{enumerate}

Иерархическая структура карты сильно ускоряет поиск пути, однако алгоритм HPA* работает не учитывая такие важные параметры как размер агентов и проходимость местности.  

В итоге размер агентов и проходимость карты должны учитываться при нахождении пути, что в случае с алгоритмом HPA* приводит к тому, что эти параметры должны учитываться при оценки путей между входами.

Алгоритм иерархического поиска является достаточно абстрактным для учёта указанных проблем. Для этого на основе алгоритма HPA* был создан алгоритм HAA* (Hierarchical Annotated A*), который при создании путей между входами кластера учитывает размеры атёров и проходимость местности.

Основная разница с алгоритмом HPA* у алгоритма HAA* состоит в шаге формирования пути между транзитными точками в рамках кластера и дальнейшем шаге их оптимизации. При нахождении пути между транзитными точками в кластере следует найти пути для всех агентов разных размеров и проходимости 

В итоге алгоритм HAA* имеет такие же преимущества как HPA*, а так же возможность учёта размера агента и проходимости карты. В зависимости от выбранной тактики устранения похожих путей между транзитными точками кластеров конечный путь будет хуже оптимального до 4-8\%.

По сравнению с JPS алгоритмы HPA* и HAA*, работают дольше и являются намного сложнее в реализации, которая может быть несоразмерна с полученной от них выгодой, однако они могут выдавать начальные участки пути намного быстрее чем JPS и A*, что в некоторых случаях оправдывает их написание.

Для реализации в выпускной работе были выбраны алгоритмы A* и JPS.  

\subsection{Различные представления области поиска}



\subsection{Эвристические функции}



\subsection{Постановка задачи}

Целью выпускной работы является проведение анализа существующих алгоритмов поиска путей, разработка различных вариантов алгоритмов и их оптимизаций, создание оптимизированной универсальной библиотеки для нахождения оптимальных маршрутов и анализ полученных результатов. Создание библиотеки состоит из заданий:

\begin{enumerate}
    \item анализ алгоритмов нахождения путей;
    \item разработка алгоритма A*;
    \item разработка алгоритма JPS;
    \item разработка препроцессинга GoalBounding;
    \item интеграция GoalBounding в алгоритм A*;
    \item интеграция GoalBounding в алгоритм JPS;
    \item разработка визуализатора для наглядной оценки и проверки алгоритмов;
    \item создание модуля для сравнения и валидации алгоритмов;
\end{enumerate}

Для написания библиотеки был выбран язык C++. Для визуализации результатов была выбрана библиотека SFML и SFGUI. Для использования параллельности - библиотека threadpool11.

\clearpage

\section{\MakeTextUppercase{ПРОЕКТИРОВАНИЕ ПРОГРАММНОГО ОБЕСПЕЧЕНИЯ}}
\subsection{Программное обеспечение}

Для написание библиотеки нахождения кратчайших путей был выбран язык программирования C++ стандарта C++11  

\subsection{Архитектура ПО}
\subsection{UML-моделирование ПО}



\clearpage

\section[Описание программной реализации]{\MakeTextUppercase{ОПИСАНИЕ ПРОГРАММНОЙ РЕАЛИЗАЦИИ}}

\vspace{1\baselineskip} 

При реализации библиотеки и визуализатора возникли задачи не связанные с реализацией самих алгоритмов, такие как: корректное измерение времени работы алгоритмов и включение отдельных методов в зависимости от типа с которым работает алгоритм.

Для корректного измерение времени был создан класс MeasureUtils, который включает в себя методы для измерения скорости работы обычных функций и методов классов. Данные методы являются шаблонными, что позволяет добиться их универсальности. Они принимают настройки тестирования, которые содержат количество вызовов функции для прогрева и количество вызовов для измерения скорости, так же передаётся сама функция и её аргументы. Прогрев нужен что бы данные и инструкции гарантировано оказались в кэше процессора, чем уменьшили влияние времени на пересылку данных из оперативной памяти и кэшей нижнего уровня. Время выполнения функции равно общему времени выполнения её в цикле поделённому на количество итераций в цикле.

Включение и выключение методов в данной работе требуется для работы со знаковы и беззнаковыми целыми типами - для беззнаковых типов не требуется проверка на то, является ли переменная отрицательной, что нужно для проверки находится ли точка в границах карты. Данная проблемы решена двумя макросами  ``PF\_FUN\_ENABLE\_IF\_SIGNED'' и ``PF\_FUN\_ENABLE\_IF\_UNSIGNED'' \cref{fig:enable_fun_macro}.

\addimg{img/enable_fun_macro.png}{1}{Макросы для включения методов}{enable_fun_macro}

\subsection{Реализации алгоритма A*}

Алгоритм A* является базовым и наиболее часто используемым алгоритмом для поиска путей. Для работы алгоритм использует открытый и закрытый список. В открытый список добавляется начальная точка, открытый список содержит точки, которые мы уже нашли, но ещё не рассмотрели. После чего пока в открытом списке существуют точки -- выбирается точка с наименьшей стоимостью и рассматриваются её соседи. Если стоимость пути через данную точку до соседа меньше записанной в него стоимости, то стоимость пути до него изменяется и его родитель меняется на текущую клетку \cref{fig:a_star_open_list_add}. После рассмотрения точки - она добавляется в закрытый список. Если открытый список оказался пуст -- значит пути до конечной точки не существует. Если из открытого списка была взята конечная точка - это означает, что путь существует и можно реконструировать его. Для этого мы берём родителя конечной точки и рекурсивно рассматриваем её родителей в массив, пока не дойдём до начальной точки \cref{fig:a_star_reconstruct_path}.

\addimg{img/a_star_open_list_add.png}{1}{Добавление в открытый список в A*}{a_star_open_list_add}

\addimg{img/a_star_reconstruct_path.png}{1}{Реконструкция пути}{a_star_reconstruct_path}

При написании A* для открытого списка была использована очередь с приоритетом std::priority\_queue, а для закрытого списка - вектор имеющий размер карты. Такая реализация закрытого списка дала возможность делать проверку на вхождение в него за константное время. Память для закрытого списка выделяется один раз при создании экземпляра алгоритма поиска и при каждом поиске у всех клеток сбрасывается флаг посещённости.

\subsection{Реализация алгоритма JPS}

Алгоритм JPS представляет собой усовершенствование алгоритма A*. Недостатком A* является то, что он добавляет в открытый список все клетки, которые являются прямыми соседями рассматриваемой и имеют больший вес чем вес текущая клетка плюс стоимость пути между ними, JPS в свою очередь предоставляет возможность пропуска добавления многих клеток на основе возможной симметрии путей. Множество симметричных путей возникает на отрытых пространствах. Пути называются симметричными, потому что они практически идентичны. Алгоритм A* не учитывает возможность симметричности путей, тогда как JPS использует её вводя некоторые допущения о карте для увеличения производительности поиска. 

Для пропуска добавления лишних клеток вводятся функции прыжков, которые делятся на горизонтальные, вертикальные и диагональные.

Логика работы горизонтального и вертикального прыжка одинакова. Рассмотрим горизонтальный прыжок право - остальные варианты происходят по аналогии с ним. При этом мы можем сделать следующие допущения \cite{JPS_DETAILS}:

\begin{enumerate}
    \item клетка из который мы пришли может быть проигнорирована;
    \item клетки по диагонали позади рассматриваемой, мы тоже может игнорировать, так как они были достигнуты из родительской клетки;
    \item клетки выше и ниже рассматриваемой могут быть достигнуты оптимальнее из её родителя;
    \item клетки правее и выше/ниже рассматриваемой могут быть достигнуты оптимальнее из клеток на одну левее;
\end{enumerate}

Эти допущения приводят к тому что алгоритм должен рассматривать только клетки правее от текущий пока путь не содержит препятствий. Однако путь не всегда свободен от препятствий, что ломает приведённой допущение. Это происходит в том случае, когда клетка сверху или снизу рассматриваемой является препятствием, что делает утверждение о том, что оптимальный путь до диагональной клетки не проходит через рассматриваемую. В такой ситуации прыжок останавливается и клетка по диагонали (такая клетка называется вынужденным соседом) и текущая клетка добавляются в открытый список для дальнейшего рассмотрения. 

Последним допущением является то, что если при прыжке препятствие блокирует продвижение в заданном направлении - весь прыжок может быть отброшен.

Такие же правила и допущения верны для диагональных прыжков. Рассмотрим прыжок вправо вверх по диагонали, можно предположить, что клетки снизу, снизу справа, слева и слева сверху можно оптимально достичь через родителя рассматриваемой клетки. В следствии  чего остаётся рассмотреть три клетки: сверху, справа и по диагонали. В отличии от горизонтального и вертикального прыжка в данном случае осталось три клетки для рассмотрения, однако для двух из них требуется вертикальный и горизонтальный прыжок. Так как требуемые прыжки уже определены - вначале происходят они, затем если в результате прыжков не было найдено клеток для дальнейшего рассмотрения - происходит прыжок по диагонали на одну клетку и процесс повторяется.

Когда прыжок закончен берётся клетка с наименьшим весом из открытого списка и из неё происходит прыжок в направлении в котором алгоритм пришёл в неё.

JPS начинается с того, что из начальной точки происходят прыжки во все восемь сторон. Затем выполняется цикл по открытому списку и если в нём существует клетка, то из неё происходит вертикальный и горизонтальный прыжок, затем выполняется проверка не допускающая прохождение между двумя непроходимыми клетками. После чего запускается диагональный прыжок. 

Диагональный прыжок происходит пока следующая клетка проходима, не является конечной клеткой и находится на карте. Пусть текущая клетка имеет координаты ${(x_c, y_c)}$, а направлением прыжка является пара дельт ${(d_x, d_y)}$ принимающих значения ${d_x, d_y \in \{-1, 0, 1\}}$. Сперва рассматривается соседняя клетка ${(x_c, y_c+d_y)}$ и если она проходима и соседняя  ${(x_c-d_x, y_c)}$ не проходима, а клетка с координатами ${(x_c-d_x, y_c+d_y)}$ проходима (путь не заблокирован), то текущая клетка проверяется на возможность добавления в открытый список. Такая же проверка происходит с тройкой точек ${\{(x_c+d_x), (x_c, y_c-d_y), (x_c+d_x, y_c-d_y)\}}$ соответственно. Если прыжок вперёд заблокирован двумя клетками по горизонтали и вертикали ${\{(x_c, y_c+d_y), (x_c+d_x, y_c)\}}$, то прыжок прекращается, иначе проводится горизонтальный и вертикальный прыжок и если хотя бы один из них нашёл клетку для дальнейшего рассмотрения, то текущая клетка добавляется в открытый список.
  
Горизонтальный и вертикальный прыжки происходят пока следующая клетка проходима, не является конечной клеткой и находится на карте. На каждом шаге происходит проверка на наличие вынужденных соседей и если такой сосед найден, то он добавлется в октрытый список и метод прерывается. 

Когда открытый список иcчерпан - происходит восстановление пути так же как и в A*.

\subsection{Реализация и интеграция GoalBounds}

Алгоритм Goal Bounding можно разделить на два этапа: этап препроцессинга и проверка направления во время выполнения поиска пути. Во время препроцессинга происходит прохождение по всем клеткам карты, для каждой клетки выполняет волновой алгоритм, который заполняет все клетки стоимостью путей до них от начальной клетки и изначальным направлением по которому алгоритм пришёл в данную клетку. После чего для данной клетки определяется 8 ограничивающих прямоугольников \cref{fig:goal_bounding_dbg}, по одному на каждое направление \cref{fig:goal_bounding_fill}. Определяются они по такому алгоритму: 

\begin{enumerate}
	\item делаем минимальную точку прямоугольника равную координатам правой верхней точки карты, а максимальную - равной левой нижней;
	\item идём по всем точкам и расширяем прямоугольник соответствующий направлению в котором точка была достигнута из начальной, что бы прямоугольник включал её. 
\end{enumerate}

\addimg{img/goal_bounding_fill.png}{1}{Определение ограничивающих прямоугольников}{goal_bounding_fill}

\addimg{img/goal_bounding_dbg.png}{0.8}{Часть карты с раскрашенными в разные цвета ограничивающими прямоугольниками}{goal_bounding_dbg}

Так как алгоритм вычисляет ограничивающие прямоугольники для каждой клетки независимо, то его можно распараллелить. Для этого был использован пул поток, реализованный библиотекой threadpool11. Для каждой клетки создаётся своя задача и добавляется в очередь, после добавления всех задач происходит ожидание их завершения. Каждый поток имеет свою карту для волнового алгоритма которая создаётся один раз для одного потока. После исполнения алгоритма результат записывается в файл рядом с файлом карты, в файле поочерёдно, для каждой клетки, записано 32 целочисленных значения - четыре значения для каждого из восьми направлений из клетки, которые являются верхней левой и правой нижней точкой прямоугольника. Если файл с результатами вычислений уже существует на диске, то загружается он, что бы не проводить трудоёмкие вычисления заново.

Для включения Goal Bounding в A* требуется добавить проверку на правильность направления сразу после проверки на вхождение точки в границу карты при циклическом обходе соседей текущей точки \cref{fig:a_star_test_goal_bounds}. 

Для добавления Goal Bouding в JPS требуется провести аналогичное действие, но для всех функций прыжков.

\addimg{img/a_star_test_goal_bounds.png}{1}{Проверка Goal Bounds}{a_star_test_goal_bounds}

\subsection{Реализация визуализатора}

Для реализации визуализатора были использованы кроссплатформенные библиотеки SFML и SFGUI. Карта рисуется несколькими слоями, которые для удобства и скорости отрисовываются в отдельные текстуры. Визуализатор рисует такие слои: слой карты, последний найденный путь, открытые клетки алгоритмом A*, направления прыжков алгоритма JPS, ограничивающие прямоугольники алгоритма Goal Bounding для выбранной клетки.

Нахождение пути происходит после выбора начальной и конечной точки, после чего он отображается на карте и выводится приблизительное время выполнения алгоритма. 

Визуализатор, кроме отображения путей с отладочной информацией имеет возможность сравнивать скорость выполнения алгоритмов для выбранного пути.

\clearpage

\section[Анализ результатов]{\MakeTextUppercase{АНАЛИЗ РЕЗУЛЬТАТОВ}}

\subsection{Сравнительный анализ алгоритмов}

\subsection{Возможные дальнейшие улучшения}

%\clearpage
%
%\input{content/testing}

\clearpage

\section*{ВЫВОДЫ}
\addcontentsline{toc}{section}{Выводы}

\vspace{1\baselineskip} 

В ходе аттестационной работы был проведён анализ существующих алгоритмов нахождения пути, способов представления области поиска и эвристик. На основе анализа областью представления была выбрана квадратная сетка, выбранными алгоритмами поиска стали A*, JPS и их модификация Goal Bounding.

В ходе выполнения работы была разработана архитектура библиотеки и интерфейса взаимодействия с ней. Была создана диаграмма вариантов использования и классов, а так же были определены программные требования. Были выбраны следующие технологии и библиотеки для создания программного продукта: C++11, threadpool11, SFML, SFGUI.

Разработана библиотека, которая реализует выбранные алгоритмы, визуализатор для анализа эффективности функций библиотеки и программа для тестирования и сравнения алгоритмов. Библиотека может быть легко интегрирована с проектами, в которых необходимо находить кратчайшие пути.

Функции библиотеки были протестированы на более чем миллионе сценариев включающих 267 различных карт, в процессе чего были устранены найденные в алгоритмах ошибки.

Выполнен анализ функций библиотеки. Определена зависимость времени поиска от стоимости пути. Анализ показал, что алгоритм A* без модификаций не подходит для нахождения пути на больших картах, модификация Gaol Bounding ускоряет A* на длинных путях в несколько раз и делает его пригодным в некоторых случаях, в свою очередь алгоритм JPS постоянно быстрее A* в 5 -- 10 раз и в несколько раз быстрее комбинации A* и Goal Bounding. Комбинация алгоритмов JPS и Goal Bounding дают ещё большее ускорение -- до одного порядка. В результате если карта является неизменной, то JPS с Goal Bounding предоставляет самую большую скорость выполнения, иначе стоит выбрать просто JPS.

Разработанная библиотека для нахождения путей является актуальной в связи с потребностью оптимизации рассмотренных алгоритмов. Библиотека для нахождения путей в игровых приложениях решает проблему скорости поиска путей на квадратной сетке.

\clearpage

\section*{СПИСОК ЛИТЕРАТУРЫ}
\addcontentsline{toc}{section}{Список литературы}

\vspace{1\baselineskip} 

\begin{thebibliography}{9}
    
    \the\citetoks
    
\end{thebibliography}

%
%1. John L. Viescas. SQL Queries for Mere Mortals. [Текст] /  John L. Viescas, Michael J. Hernandez, 2007. - 672с.
%
%2. David Geary. Core JavaServer Faces (3rd Edition). [Текст] / David Geary, Cay S. Horstmann, 2010. - 672c.
%
%3. James Elliott. Harnessing Hibernate. [Текст] / James Elliott, Timothy M. O'Brien, Ryan Fowler, 2008. - 382c.
%
%4. Oleg Varaksin. PrimeFaces Cookbook. [Текст] / Oleg Varaksin, Mert Caliskan, 2013. - 328c.

%\clearpage
%
%\section*{ПРИЛОЖЕНИЕ A \\
	Описание предприятия }
\addcontentsline{toc}{section}{Приложение A Описание предприятия}


\vspace{1\baselineskip} 

Университет ХНУРЭ - технический университет в Харькове. 


В университете обучается более 12 тысяч студентов по 34 специальностям.



%\clearpage
%
%\section*{ПРИЛОЖЕНИЕ A \\
	Описание предприятия }
\addcontentsline{toc}{section}{Приложение A Описание предприятия}


\vspace{1\baselineskip} 

Университет ХНУРЭ - технический университет в Харькове. 


В университете обучается более 12 тысяч студентов по 34 специальностям.



\end{document}
