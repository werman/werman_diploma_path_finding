\section*{ВВЕДЕНИЕ}
\addcontentsline{toc}{section}{Введение}

\vspace{1\baselineskip} 

Задачей нахождения кратчайшего пути является поиск оптимального и короткого пути между двумя точками. Проблема нахождения кратчайших путей возникает в таких случаях как: оптимизация перевозки грузов и пассажиров, оптимальная маршрутиризация пакетов в сети, навигация искусственного интеллекта и игрока в компьютерных играх, а так же навигация роботов в пространстве. Все компьютерные игры имеют поиск путей в том или ином виде, поэтому скорость и точность алгоритма часто влияет на качество искусственного интеллекта и восприятия игрока. 

Существует несколько представлений пространства для проведения поиска путей, одним из них является квадратная сетка, которая проста для создания и используется в стратегиях реального времени и других играх с двумерной картой. Хотя квадратная сетка в большинстве случаев является неоптимальным представлением области поиск, с ней очень просто работать и легко модифицировать, что значительно упрощает программную работу с игровой картой. В следствии неоптимальности представления карты появляется необходимость в оптимизации алгоритмов поиска работающих с ней посредством общей оптимизации логики работы алгоритма, введение некоторых допущений и ограничений на область поиска, а так же проведение низкоуровневых оптимизаций.

Таким образом, задача нахождения кратчайшего пути на области представленной квадратной сеткой является актуальной проблемой, которая будет рассмотрена и исследована в данной работе.

Целью преддипломной практики является проведение анализа существующих алгоритмов поиска путей, разработка различных вариантов алгоритмов и их оптимизаций, создание оптимизированной универсальной библиотеки для нахождения оптимальных маршрутов и анализ полученных результатов.

