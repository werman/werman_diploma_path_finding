\section*{ВВЕДЕНИЕ}
\addcontentsline{toc}{section}{Введение}

\vspace{1\baselineskip} 

Задачей нахождения кратчайшего пути является поиск оптимального и короткого пути между двумя точками. Проблема нахождения кратчайших путей возникает в таких случаях как: оптимизация перевозки грузов и пассажиров, оптимальная маршрутиризация пакетов в сети, навигация искусственного интеллекта и игрока в компьютерных играх, а так же навигация роботов в пространстве. На данный момент большинство компьютерных игр имеют поиск путей в том или ином виде, поэтому скорость и точность алгоритма часто влияет на качество искусственного интеллекта и восприятие игры игроком. 

Разнообразие жанров компьютерных игор приводит к разнообразию представлений карт с которыми приходится работать алгоритмам поиска путей. Одним из таких представлений является квадратная сетка, которая часто используется в играх с двумерной картой, например игры жанра RTS. Хотя квадратная сетка в большинстве случаев является неоптимальным представлением области поиск, с ней очень просто работать и легко модифицировать, что значительно упрощает программную работу с игровой картой. В следствии неоптимальности представления карты появляется необходимость в оптимизации алгоритмов поиска работающих с ней посредством общей оптимизации логики работы алгоритма, введение некоторых допущений и ограничений на область поиска, а так же проведение низкоуровневых оптимизаций.

Одними из методов нахождения кратчайшего пути являются алгоритмы A* и JPS, которые широко используются в игровых приложениях. Данные алгоритмы работаю без предварительных расчётов, однако такие расчёты могут ускорить поиск пути при определённых условиях \cite{PREPROCESSING}. 

Таким образом, задача нахождения кратчайшего пути на области представленной квадратной сеткой является актуальной проблемой, которая будет рассмотрена и исследована в данной работе.

Целью выпускной работы является проведение анализа существующих алгоритмов поиска путей, разработка различных вариантов алгоритмов и их оптимизаций, создание оптимизированной универсальной библиотеки для нахождения оптимальных маршрутов и анализ полученных результатов.

