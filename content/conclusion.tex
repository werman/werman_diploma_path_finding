\section*{ВЫВОДЫ}
\addcontentsline{toc}{section}{Выводы}

\vspace{1\baselineskip} 

Я, Пиляев Данил Викторович, проходил преддипломною практику на базе университета ХНУРЭ  в период с 18 апреля 2016 р. по 14 мая 2016р. Во время прохождения практики я получил опыт в проектировании программных систем в теории и на практике, получил новые знания по разработке и тестированию программного обеспечения. В ходе практики были углублены знания языка C++ и приобретены знания по оптимизации алгоритмов.

В результате работы был проведён анализ существующих алгоритмов нахождения пути, способов представления области поиска и эвристик. Областью представления была выбрана квадратная сетка, выбранными алгоритмами поиска стали A* и JPS с модификацией Goal Bounding.

В результате работы была разработана архитектура библиотеки и интерфейса взаимодействия с ней. Была создана диаграмма вариантов использования и классов, а так же были определены программные требования. Были выбраны следующие технологии и библиотеки для создания программного продукта: C++11, threadpool11, SFML, SFGUI.

Разработанная библиотека для нахождения путей является актуальной в связи с потребностью оптимизации рассмотренных алгоритмов. Библиотека для нахождения путей в игровых приложениях решает проблему скорости поиска путей на квадратной сетки.