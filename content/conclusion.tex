\section*{ВЫВОДЫ}
\addcontentsline{toc}{section}{Выводы}

\vspace{1\baselineskip} 

В ходе аттестационной работы был проведён анализ существующих алгоритмов нахождения пути, способов представления области поиска и эвристик. На основе анализа областью представления была выбрана квадратная сетка, выбранными алгоритмами поиска стали A*, JPS и их модификация Goal Bounding.

В ходе выполнения работы была разработана архитектура библиотеки и интерфейса взаимодействия с ней. Была создана диаграмма вариантов использования и классов, а так же были определены программные требования. Были выбраны следующие технологии и библиотеки для создания программного продукта: C++11, threadpool11, SFML, SFGUI.

Разработана библиотека, которая реализует выбранные алгоритмы, визуализатор для анализа эффективности функций библиотеки и программа для тестирования и сравнения алгоритмов. Библиотека может быть легко интегрирована с проектами, в которых необходимо находить кратчайшие пути.

Функции библиотеки были протестированы на более чем миллионе сценариев включающих 267 различных карт, в процессе чего были устранены найденные в алгоритмах ошибки.

Выполнен анализ функций библиотеки. Определена зависимость времени поиска от стоимости пути. Анализ показал, что алгоритм A* без модификаций не подходит для нахождения пути на больших картах, модификация Gaol Bounding ускоряет A* на длинных путях в несколько раз и делает его пригодным в некоторых случаях, в свою очередь алгоритм JPS постоянно быстрее A* в 5 -- 10 раз и в несколько раз быстрее комбинации A* и Goal Bounding. Комбинация алгоритмов JPS и Goal Bounding дают ещё большее ускорение -- до одного порядка. В результате если карта является неизменной, то JPS с Goal Bounding предоставляет самую большую скорость выполнения, иначе стоит выбрать просто JPS.

Разработанная библиотека для нахождения путей является актуальной в связи с потребностью оптимизации рассмотренных алгоритмов. Библиотека для нахождения путей в игровых приложениях решает проблему скорости поиска путей на квадратной сетке.