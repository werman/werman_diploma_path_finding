\section{\MakeTextUppercase{ПРОЕКТИРОВАНИЕ ПРОГРАММНОГО ОБЕСПЕЧЕНИЯ}}
\subsection{Программное обеспечение}

Для написание библиотеки нахождения кратчайших путей был выбран язык программирования C++ стандарта C++11. С++ является современным языком высокого уровня, который предоставляет широкие возможности по оптимизации кода, в отличии от интерпретируемых языков и языков с JIT оптимизациями. Так же С++ широко используется в игровых приложениях. Для создания библиотеки язык С++ был выбран по следующим причинам: кроссплатформенность, быстродействие, низкоуровневые оптимизаций, возможность внедрения библиотеки в существующие программные продукты. 

Стандарт C++11 привнёс в язык многие функции, которые позволяют писать более понятный и современный код, уменьшить количество случайных ошибок и повысить читаемость. В следствии чего были устранены многие недостатки в сравнении с другими языками программирования.

В C++11 для измерения времени используется стандартная библиотека chrono. В chrono существует несколько реализаций измерения времени:

\begin{enumerate}
    \item system\_clock - общесистемное время;
    \item steady\_clock - монотонное время, которое никогда не подстроено;
    \item high\_resolution\_clock - наиболее точное время с наименьшим доступным периодом.  
\end{enumerate} 

Библиотека SFML (Simple and Fast Multimedia Library) - кроссплатформенная библиотека для создания мультимедийных приложений. Имеет простой платформонезависимый интерфейс для рисования графики.

SFGUI - библиотека работающая совместно с библиотекой SFML и предоставляющая возможности для отрисовывания интерфейсов.

Для использования многопоточности была подключена библиотека threadpool11, которая реализует пул потоков. Пул потоков -- 

Для написания кода библиотеки была выбрана среда разработки CLion, которая имеет редактор с поддержкой синтаксиса C++11 и его подсветкой, имеет средства рефактоинга и поддерживает различные CVS, например Git. Так же имеется поддержка CMake - системы кроссплатформенной сборки проектов и управления зависимостями и тестами.

В качестве системы контроля версий был выбран Git. Git - распределённая система контроля версий, которая направлена на скорость работа и целостность данных, имеет гибкую и простую систему создания и объединения веток.

\subsection{Архитектура ПО}



\subsection{UML-моделирование ПО}

Унифицированный язык моделирования (UML) - язык общего назначения для визуализации, спецификации, конструирования и документации программных систем \cite{UML_USER_GUIDE_2ND}. Язык UML объединяет в себе семейство разных графических нотаций с общей метамоделью. 

Преимуществами UML являются:

\begin{enumerate}
    \item объектно-ориентированность, что делает его близким к современным объектного-ориентированным языкам;
    \item расширяем, что позволяет вводить собственные текстовые и графические стереотипы;
    \item прост для чтения;
    \item позволяет описать системы со всех точек зрения.
\end{enumerate}

В UML используется три вида диаграмм:

\begin{enumerate}
    \item структурные диаграммы - отражают статическую структуру системы;
    \item диаграммы поведения - отражают поведение системы в динамике, показывают, что должно происходить в системе;
    \item диаграммы взаимодействия - подвид диаграмм поведения, которые выражают передачу контроля и данных внутри системы.
\end{enumerate} 

Для моделирования системы проектируемой в данной выпускной работе будут использованы структурные диаграммы, а именно диаграмма вариантов использования (Use case diagram) и диаграмма классов (Class diagram).

Диаграмма вариантов использования - 

Диаграмма классов - 


\subsection{Описание интерфейса взаимодействия}

