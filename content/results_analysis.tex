\section[Анализ результатов]{\MakeTextUppercase{АНАЛИЗ РЕЗУЛЬТАТОВ}}

\subsection{Сравнительный анализ алгоритмов}

\subsection{Возможные дальнейшие улучшения}

В ходе выполнения работы было

Одним из недостатков алгоритма JPS является то, что он работает только на картах с одинаковой стоимостью прохождения по клеткам. Однако в играх часто имеется потребность в неоднородных картах, что исключает исспользование данного алгоритма без дополнительных модификаций. В качестве модификации для поддержки неоднородных карт может быть рассмотрено следующие предположение: если на однородной карте вынужденные соседи возникали когда клетку преграждало препятствие, то на неоднородной карте как препятствие можно так же рассматривать смену стоимости прохождения через клетку. При этом если стоимости прохождения по клеткам для соседних клеток в большинстве случаев различны, то JPS теряет своё преимущество перед A*. Для исправления такой ситуации можно уменьшить разрешение стоимости пути, что бы у соседних клеток чаще была одинаковая стоимость. Или ввести некоторую адаптивную дельту если разность стоимости пути между клетками меньше её, то они считаются одинаковыми. Данная модификация является чисто теоритической, так как никем не была реализована на практике.

Алгоритм Goal Bounding позволяет существенно ускорить A* и JPS, однако при этом часть препроцессинга выполняется очень долго, результат занимает много места и не поддерживает изменение карты во время выполнения. 