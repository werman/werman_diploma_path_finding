\section*{РЕФЕРАТ}

\vspace{1\baselineskip}

\thispagestyle{empty}

Пояснювальна записка: \pageref{LastPage} сторінок, \totalsections\ розділів,  \totalfigures\ рисунків, \totaltables\ таблиця, 13 джерел, 3 додатки.

Об'єкт проектування -- програма для пошуку найкоротших шляхів.

Метою роботи є створення оптимізованної бібліотеки для пошуку найкоротших шляхів в ігрових додатків.

Методи розробки базуються на мові С++11, та бібліотеках threadpool11, SFML і SFGUI

У результаті роботи була здійснена програмна реалізація бібліотеки для пошуку шляхів в ігрових додатків, а саме алгоритмів A*, JPS та Goal Bounding. Виконано оптимізацію алгоритмів за
часовими характеристиками.

БИБЛИОТЕКА, ПОИСК ПУТЕЙ, C++, A*, JPS, GOAL BOUNDING, HAA*, КВАДРАТНАЯ СЕТКА, ЭВРИСТИЧЕСКАЯ ФУНКЦИЯ.

\vspace{1\baselineskip}

Explanatory note: \pageref{LastPage} pages, \totalsections\ sections,  \totalfigures\ figures, \totaltables\ table, 13 sources, 3 applications.

Subject of architecturing -- library for path finding.

The aim is to create optimized library for path finding in games.

The development methods are based on С++11 and libraries threadpool11, SFML and SFGUI.

The result of work is implemented library for path finding in games which includes A* and JPS algorithms. During development this algorithms were analyzed and optimized.

LIBRARY, ПPATH FINDING, C++, A*, JPS, GOAL BOUNDING, HAA*, SQUARE GRID, HEURISTIC FUNCTION.