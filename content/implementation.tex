\section[Описание программной реализации]{\MakeTextUppercase{ОПИСАНИЕ ПРОГРАММНОЙ РЕАЛИЗАЦИИ}}

\vspace{1\baselineskip} 

При реализации библиотеки и визуализатора возникли задачи не связанные с реализацией самих алгоритмов, такие как: корректное измерение времени работы алгоритмов и включение отдельных методов в зависимости от типа с которым работает алгоритм.

Для корректного измерение времени был создан класс MeasureUtils, который включает в себя методы для измерения скорости работы обычных функций и методов классов. Данные методы являются шаблонными, что позволяет добиться их универсальности. Они принимают настройки тестирования, которые содержат количество вызовов функции для прогрева и количество вызовов для измерения скорости, так же передаётся сама функция и её аргументы. Прогрев нужен что бы данные и инструкции гарантировано оказались в кэше процессора, чем уменьшили влияние времени на пересылку данных из оперативной памяти и кэшей нижнего уровня. Время выполнения функции равно общему времени выполнения её в цикле поделённому на количество итераций в цикле.

Включение и выключение методов в данной работе требуется для работы со знаковы и беззнаковыми целыми типами - для беззнаковых типов не требуется проверка на то, является ли переменная отрицательной, что нужно для проверки находится ли точка в границах карты. Данная проблемы решена двумя макросами  ``PF\_FUN\_ENABLE\_IF\_SIGNED'' и ``PF\_FUN\_ENABLE\_IF\_UNSIGNED'' \cref{fig:enable_fun_macro}.

\begin{figure}[!htb]
	\centering
	\captionsetup{justification=centering}
	\begin{lstlisting}
#define PF_FUN_ENABLE_IF_SIGNED( ValueType, ReturnType ) \
template<class Q = ValueType> \
enable_if_t<std::is_integral<Q>::value &&  std::is_signed<Q>::value, ReturnType>

#define PF_FUN_ENABLE_IF_UNSIGNED( ValueType, ReturnType ) \
template<class Q = ValueType> \
enable_if_t<std::is_integral<Q>::value &&  std::is_unsigned<Q>::value, ReturnType>
	\end{lstlisting}
	\caption{Макросы для включения методов}
	\label{fig:enable_fun_macro}
\end{figure}

\subsection{Реализации алгоритма A*}

Алгоритм A* является базовым и наиболее часто используемым алгоритмом для поиска путей. Для работы алгоритм использует открытый и закрытый список. В открытый список добавляется начальная точка, открытый список содержит точки, которые мы уже нашли, но ещё не рассмотрели. После чего пока в открытом списке существуют точки -- выбирается точка с наименьшей стоимостью и рассматриваются её соседи. Если стоимость пути через данную точку до соседа меньше записанной в него стоимости, то стоимость пути до него изменяется и его родитель меняется на текущую клетку \cref{fig:a_star_open_list_add}. После рассмотрения точки - она добавляется в закрытый список. Если открытый список оказался пуст -- значит пути до конечной точки не существует. Если из открытого списка была взята конечная точка - это означает, что путь существует и можно реконструировать его. Для этого мы берём родителя конечной точки и рекурсивно рассматриваем её родителей в массив, пока не дойдём до начальной точки \cref{fig:a_star_reconstruct_path}.

\begin{figure}[!htb]
	\centering
	\captionsetup{justification=centering}
	\begin{lstlisting}
NodeInfo& nextNodeInfo = m_nodesInfo[nextNodeIdx];
if( !nextNodeInfo.IsVisited() || newCost < nextNodeInfo.GetNodeCost())
{
	nextNodeInfo.SetVisited( true );
	nextNodeInfo.SetNodeCost( newCost );
	nextNodeInfo.SetCameFrom( best.x, best.y );
	float priority =
		newCost +
		Heuritic::nodes_distance( nextX, nextY, targetX, targetY, startX, startY );
	openList.push( PriorityNode<CoordsType>( nextX, nextY, priority ));
}
	\end{lstlisting}
	\caption{Добавление в открытый список в A*}
	\label{fig:a_star_open_list_add}
\end{figure}

\begin{figure}[!htb]
	\centering
	\captionsetup{justification=centering}
	\begin{lstlisting}
std::vector<Point<CoordsType>> path;
auto current = Point<CoordsType>( targetX, targetY );
path.push_back( current );
while( !current.Equals( startPoint ))
{
	auto nodeIdx = coords_to_contiguous_idx( current.x, current.y, m_map.GetSizeX());
	const auto& info = m_nodesInfo[nodeIdx];
	if( current == info.GetCameFrom())
	{
		break;
	}
	
	current = info.GetCameFrom();
	path.push_back( current );
}
	\end{lstlisting}
	\caption{Реконструкция пути}
	\label{fig:a_star_reconstruct_path}
\end{figure}

При написании A* для открытого списка была использована очередь с приоритетом std::priority\_queue, а для закрытого списка - вектор имеющий размер карты. Такая реализация закрытого списка дала возможность делать проверку на вхождение в него за константное время. Память для закрытого списка выделяется один раз при создании экземпляра алгоритма поиска и при каждом поиске у всех клеток сбрасывается флаг посещённости.

\subsection{Реализация алгоритма JPS}

Алгоритм JPS представляет собой усовершенствование алгоритма A*. Недостатком A* является то, что он добавляет в открытый список все клетки, которые являются прямыми соседями рассматриваемой и имеют больший вес чем вес текущая клетка плюс стоимость пути между ними, JPS в свою очередь предоставляет возможность пропуска добавления многих клеток на основе возможной симметрии путей. Множество симметричных путей возникает на отрытых пространствах. Пути называются симметричными, потому что они практически идентичны. Алгоритм A* не учитывает возможность симметричности путей, тогда как JPS использует её вводя некоторые допущения о карте для увеличения производительности поиска. 

Для пропуска добавления лишних клеток вводятся функции прыжков, которые делятся на горизонтальные, вертикальные и диагональные.

Логика работы горизонтального и вертикального прыжка одинакова. Рассмотрим горизонтальный прыжок право - остальные варианты происходят по аналогии с ним. При этом мы можем сделать следующие допущения \cite{JPS_DETAILS}:

\begin{enumerate}
    \item клетка из который мы пришли может быть проигнорирована;
    \item клетки по диагонали позади рассматриваемой, мы тоже можем игнорировать, так как они были достигнуты из родительской клетки \cref{fig:jps_cells_ignored_behind};
    \item клетки выше и ниже рассматриваемой могут быть достигнуты оптимальнее из её родителя \cref{fig:jps_cells_ignored_up_down};
    \item клетки правее и выше/ниже рассматриваемой могут быть достигнуты оптимальнее из клеток на одну левее;
\end{enumerate}

\addtikz{Отброшенные клетки сзади}{jps_cells_ignored_behind}
{
\mybox[x=1,y=0,color=grey]
\mybox[x=1,y=1,color=grey]
\mybox[x=1,y=2,color=grey]
\mybox[x=2,y=1,color=green]

\mygrid[width=6, height=3]

\myarrow[startx=0,starty=1,endx=1,endy=1,color=darkgrey]
\myarrow[startx=1,starty=1,endx=2,endy=1,color=darkgrey]
\myarrow[startx=1,starty=1,endx=1,endy=2,color=blue]
\myarrow[startx=1,starty=1,endx=1,endy=0,color=blue]
}

\addtikz{Отброшенные клетки сверху и снизу}{jps_cells_ignored_up_down}
{
	\mybox[x=1,y=0,color=grey]
	\mybox[x=1,y=1,color=grey]
	\mybox[x=1,y=2,color=grey]
	\mybox[x=2,y=0,color=grey]
	\mybox[x=2,y=2,color=grey]
	\mybox[x=2,y=1,color=green]
	
	\mygrid[width=6, height=3]
	
	\myarrow[startx=0,starty=1,endx=1,endy=1,color=darkgrey]
	\myarrow[startx=1,starty=1,endx=2,endy=1,color=darkgrey]
	\myarrow[startx=1,starty=1,endx=1,endy=2,color=darkgrey]
	\myarrow[startx=1,starty=1,endx=1,endy=0,color=darkgrey]
	\myarrow[startx=1,starty=1,endx=2,endy=2,color=blue]
	\myarrow[startx=1,starty=1,endx=2,endy=0,color=blue]
}

Эти допущения приводят к тому что алгоритм должен рассматривать только клетки правее от текущий пока путь не содержит препятствий \cref{fig:jps_forward_move}. Однако путь не всегда свободен от препятствий, что ломает приведённой допущение. Это происходит в том случае, когда клетка сверху или снизу рассматриваемой является препятствием, что делает утверждение о том, что оптимальный путь до диагональной клетки не проходит через рассматриваемую \cref{fig:jps_horizontal_forced_neighbour}. В такой ситуации прыжок останавливается и клетка по диагонали (такая клетка называется вынужденным соседом) и текущая клетка добавляются в открытый список для дальнейшего рассмотрения. 

Последним допущением является то, что если при прыжке препятствие блокирует продвижение в заданном направлении - весь прыжок может быть отброшен \cref{fig:jps_forward_move_blocked}.

\addtikz{Отброшенные клетки сверху и снизу}{jps_forward_move}
{
	\mybox[x=1,y=0,color=grey]
	\mybox[x=1,y=1,color=grey]
	\mybox[x=1,y=2,color=grey]
	\mybox[x=2,y=0,color=grey]
	\mybox[x=2,y=2,color=grey]
	\mybox[x=2,y=0,color=grey]
	\mybox[x=2,y=2,color=grey]
	\mybox[x=3,y=0,color=grey]
	\mybox[x=3,y=2,color=grey]
	\mybox[x=4,y=0,color=grey]
	\mybox[x=4,y=2,color=grey]
	\mybox[x=2,y=1,color=green]
	
	\mygrid[width=6, height=3]
	
	\myarrow[startx=0,starty=1,endx=1,endy=1,color=darkgrey]
	\myarrow[startx=1,starty=1,endx=2,endy=1,color=darkgrey]
	\myarrow[startx=1,starty=1,endx=1,endy=2,color=darkgrey]
	\myarrow[startx=1,starty=1,endx=1,endy=0,color=darkgrey]
	\myarrow[startx=1,starty=1,endx=2,endy=2,color=darkgrey]
	\myarrow[startx=1,starty=1,endx=2,endy=0,color=darkgrey]
	\myarrow[startx=2,starty=0,endx=3,endy=0,color=darkgrey]
	\myarrow[startx=2,starty=2,endx=3,endy=2,color=darkgrey]
	\myarrow[startx=3,starty=0,endx=4,endy=0,color=darkgrey]
	\myarrow[startx=3,starty=2,endx=4,endy=2,color=darkgrey]
	\myarrow[startx=2,starty=1,endx=3,endy=1,color=blue]
	\myarrow[startx=3,starty=1,endx=4,endy=1,color=blue]
}

\addtikz{Вынужденный сосед}{jps_horizontal_forced_neighbour}
{
	\begin{scope}[]
		\matrix[column sep=1cm, ampersand replacement=\&]{
			\mybox[x=1,y=0,color=grey]
			\mybox[x=1,y=1,color=grey]
			\mybox[x=1,y=2,color=grey]
			\mybox[x=2,y=0,color=grey]
			\mybox[x=2,y=2,color=black]
			\mybox[x=3,y=2,color=red]
			\mybox[x=3,y=0,color=grey]
			\mybox[x=2,y=1,color=green]
			
			\mygrid[width=6, height=3]
			
			\myarrow[startx=0,starty=1,endx=1,endy=1,color=darkgrey]
			\myarrow[startx=1,starty=1,endx=2,endy=1,color=darkgrey]
			\myarrow[startx=1,starty=1,endx=1,endy=2,color=darkgrey]
			\myarrow[startx=1,starty=1,endx=1,endy=0,color=darkgrey]
			\myarrow[startx=1,starty=1,endx=2,endy=2,color=red]
			\myarrow[startx=1,starty=1,endx=2,endy=0,color=darkgrey]
			\myarrow[startx=2,starty=0,endx=3,endy=0,color=darkgrey]
			\myarrow[startx=2,starty=1,endx=3,endy=2,color=blue]
			\myarrow[startx=2,starty=1,endx=3,endy=1,color=blue];
			
			\&
			
			\mybox[x=1,y=0,color=grey]
			\mybox[x=1,y=1,color=grey]
			\mybox[x=1,y=2,color=grey]
			\mybox[x=2,y=0,color=black]
			\mybox[x=2,y=2,color=grey]
			\mybox[x=3,y=2,color=grey]
			\mybox[x=3,y=0,color=red]
			\mybox[x=2,y=1,color=green]
			
			\mygrid[width=6, height=3]
			
			\myarrow[startx=0,starty=1,endx=1,endy=1,color=darkgrey]
			\myarrow[startx=1,starty=1,endx=2,endy=1,color=darkgrey]
			\myarrow[startx=1,starty=1,endx=1,endy=2,color=darkgrey]
			\myarrow[startx=1,starty=1,endx=1,endy=0,color=darkgrey]
			\myarrow[startx=1,starty=1,endx=2,endy=2,color=darkgrey]
			\myarrow[startx=2,starty=2,endx=3,endy=2,color=darkgrey]
			\myarrow[startx=1,starty=1,endx=2,endy=0,color=red]
			\myarrow[startx=2,starty=1,endx=3,endy=0,color=blue]
			\myarrow[startx=2,starty=1,endx=3,endy=1,color=blue];
			\\
		};
	\end{scope}
}

\addtikz{Заблокированный путь}{jps_forward_move_blocked}
{
	\mybox[x=1,y=0,color=grey]
	\mybox[x=1,y=1,color=grey]
	\mybox[x=1,y=2,color=grey]
	\mybox[x=2,y=0,color=grey]
	\mybox[x=2,y=2,color=grey]
	\mybox[x=2,y=0,color=grey]
	\mybox[x=2,y=2,color=grey]
	\mybox[x=3,y=0,color=grey]
	\mybox[x=3,y=2,color=grey]
	\mybox[x=4,y=0,color=grey]
	\mybox[x=4,y=2,color=grey]
	\mybox[x=2,y=1,color=green]
	\mybox[x=5,y=1,color=black]
		
	\mygrid[width=6, height=3]
	
	\myarrow[startx=0,starty=1,endx=1,endy=1,color=darkgrey]
	\myarrow[startx=1,starty=1,endx=2,endy=1,color=darkgrey]
	\myarrow[startx=1,starty=1,endx=1,endy=2,color=darkgrey]
	\myarrow[startx=1,starty=1,endx=1,endy=0,color=darkgrey]
	\myarrow[startx=1,starty=1,endx=2,endy=2,color=darkgrey]
	\myarrow[startx=1,starty=1,endx=2,endy=0,color=darkgrey]
	\myarrow[startx=2,starty=0,endx=3,endy=0,color=darkgrey]
	\myarrow[startx=2,starty=2,endx=3,endy=2,color=darkgrey]
	\myarrow[startx=3,starty=0,endx=4,endy=0,color=darkgrey]
	\myarrow[startx=3,starty=2,endx=4,endy=2,color=darkgrey]
	\myarrow[startx=2,starty=1,endx=3,endy=1,color=blue]
	\myarrow[startx=3,starty=1,endx=4,endy=1,color=blue]
}

Такие же правила и допущения верны для диагональных прыжков. Рассмотрим прыжок вправо вверх по диагонали, можно предположить, что клетки снизу, снизу справа, слева и слева сверху можно оптимально достичь через родителя рассматриваемой клетки. В следствии  чего остаётся рассмотреть три клетки: сверху, справа и по диагонали \cref{fig:jps_diagonal_jump}. В отличии от горизонтального и вертикального прыжка в данном случае осталось три клетки для рассмотрения, однако для двух из них требуется вертикальный и горизонтальный прыжок. Так как требуемые прыжки уже определены - вначале происходят они, затем если в результате прыжков не было найдено клеток для дальнейшего рассмотрения - происходит прыжок по диагонали на одну клетку и процесс повторяется.

\addtikz{Прыжок по диагонали}{jps_diagonal_jump}
{
	\mybox[x=1,y=1,color=grey]
	\mybox[x=1,y=2,color=grey]
	\mybox[x=1,y=3,color=grey]
	\mybox[x=2,y=1,color=grey]
	\mybox[x=3,y=1,color=grey]
	\mybox[x=2,y=2,color=green]
	
	\mygrid[width=5, height=5]
	
	\myarrow[startx=0,starty=0,endx=1,endy=1,color=darkgrey]
	\myarrow[startx=1,starty=1,endx=1,endy=2,color=darkgrey]
	\myarrow[startx=1,starty=2,endx=1,endy=3,color=darkgrey]
	\myarrow[startx=1,starty=1,endx=2,endy=1,color=darkgrey]
	\myarrow[startx=2,starty=1,endx=3,endy=1,color=darkgrey]
	\myarrow[startx=1,starty=1,endx=2,endy=2,color=darkgrey]
	\myarrow[startx=2,starty=2,endx=2,endy=3,color=blue]
	\myarrow[startx=2,starty=2,endx=3,endy=2,color=blue]
	\myarrow[startx=2,starty=2,endx=3,endy=3,color=blue]
}

По аналогии с горизонтальным и вертикальным прыжком при диагональном прыжке так же встречаются вынужденные соседи, когда слева или под текущей клеткой находится препятствие, клетка слева вверху или справа внизу соответственно являются вынужденными соседями \cref{fig:jps_diagonal_jump_forced_neighbour}. 

\addtikz{Вынужденный сосед при диагональном прыжке}{jps_diagonal_jump_forced_neighbour}
{
	\begin{scope}[]
		\matrix[column sep=1cm, ampersand replacement=\&]{
			% % % % % % % % % %
			\mybox[x=1,y=1,color=grey]
			\mybox[x=1,y=2,color=grey]
			\mybox[x=1,y=3,color=grey]
			\mybox[x=2,y=1,color=black]
			\mybox[x=3,y=1,color=red]
			\mybox[x=2,y=2,color=green]
			
			\mygrid[width=5, height=5]
			
			\myarrow[startx=0,starty=0,endx=1,endy=1,color=darkgrey]
			\myarrow[startx=1,starty=1,endx=1,endy=2,color=darkgrey]
			\myarrow[startx=1,starty=2,endx=1,endy=3,color=darkgrey]
			\myarrow[startx=1,starty=1,endx=2,endy=1,color=red]
			\myarrow[startx=2,starty=2,endx=3,endy=1,color=blue]
			\myarrow[startx=1,starty=1,endx=2,endy=2,color=darkgrey]
			\myarrow[startx=2,starty=2,endx=2,endy=3,color=blue]
			\myarrow[startx=2,starty=2,endx=3,endy=2,color=blue]
			\myarrow[startx=2,starty=2,endx=3,endy=3,color=blue]; 
			
			\& % % % % % % % %
			
			\mybox[x=1,y=1,color=grey]
			\mybox[x=1,y=2,color=black]
			\mybox[x=1,y=3,color=red]
			\mybox[x=2,y=1,color=grey]
			\mybox[x=3,y=1,color=grey]
			\mybox[x=2,y=2,color=green]
			
			\mygrid[width=5, height=5]
			
			\myarrow[startx=0,starty=0,endx=1,endy=1,color=darkgrey]
			\myarrow[startx=1,starty=1,endx=1,endy=2,color=red]
			\myarrow[startx=2,starty=2,endx=1,endy=3,color=blue]
			\myarrow[startx=1,starty=1,endx=2,endy=1,color=darkgrey]
			\myarrow[startx=2,starty=1,endx=3,endy=1,color=darkgrey]
			\myarrow[startx=1,starty=1,endx=2,endy=2,color=darkgrey]
			\myarrow[startx=2,starty=2,endx=2,endy=3,color=blue]
			\myarrow[startx=2,starty=2,endx=3,endy=2,color=blue]
			\myarrow[startx=2,starty=2,endx=3,endy=3,color=blue]; 
			\\
	};
	\end{scope}
}

Когда прыжок закончен берётся клетка с наименьшим весом из открытого списка и из неё происходит прыжок в направлении в котором алгоритм пришёл в неё.

JPS начинается с того, что из начальной точки происходят прыжки во все восемь сторон. Затем выполняется цикл по открытому списку и если в нём существует клетка, то из неё происходит вертикальный и горизонтальный прыжок, затем выполняется проверка не допускающая прохождение между двумя непроходимыми клетками. После чего запускается диагональный прыжок. 

Диагональный прыжок происходит пока следующая клетка проходима, не является конечной клеткой и находится на карте. Пусть текущая клетка имеет координаты ${(x_c, y_c)}$, а направлением прыжка является пара дельт ${(d_x, d_y)}$ принимающих значения ${d_x, d_y \in \{-1, 0, 1\}}$. Сперва рассматривается соседняя клетка ${(x_c, y_c+d_y)}$ и если она проходима и соседняя  ${(x_c-d_x, y_c)}$ не проходима, а клетка с координатами ${(x_c-d_x, y_c+d_y)}$ проходима (путь не заблокирован \cref{fig:jps_diagonal_jump_blocked_forced_neighbour}), то текущая клетка проверяется на возможность добавления в открытый список. Такая же проверка происходит с тройкой точек ${\{(x_c+d_x), (x_c, y_c-d_y), (x_c+d_x, y_c-d_y)\}}$ соответственно. Если прыжок вперёд заблокирован двумя клетками по горизонтали и вертикали ${\{(x_c, y_c+d_y), (x_c+d_x, y_c)\}}$ \cref{fig:jps_blocked_diagonal_jump}, то прыжок прекращается, иначе проводится горизонтальный и вертикальный прыжок и если хотя бы один из них нашёл клетку для дальнейшего рассмотрения, то текущая клетка добавляется в открытый список, при этом такие прыжки в стороны не добавляют клетки в открытый список.
  
\addtikz{Заблокированный вынужденный сосед}{jps_diagonal_jump_blocked_forced_neighbour}
{
	\mybox[x=1,y=1,color=grey]
	\mybox[x=1,y=2,color=grey]
	\mybox[x=1,y=3,color=grey]
	\mybox[x=2,y=1,color=black]
	\mybox[x=3,y=2,color=black]
	\mybox[x=3,y=1,color=yellow]
	\mybox[x=2,y=2,color=green]
	
	\mygrid[width=5, height=5]
	
	\myarrow[startx=0,starty=0,endx=1,endy=1,color=darkgrey]
	\myarrow[startx=1,starty=1,endx=1,endy=2,color=darkgrey]
	\myarrow[startx=1,starty=2,endx=1,endy=3,color=darkgrey]
	\myarrow[startx=2,starty=2,endx=2,endy=3,color=blue]
	\myarrow[startx=1,starty=1,endx=2,endy=2,color=darkgrey]
	\myarrow[startx=2,starty=2,endx=3,endy=2,color=red]
	\myarrow[startx=2,starty=2,endx=3,endy=1,color=red]
	\myarrow[startx=2,starty=2,endx=3,endy=3,color=blue]
}
  
\addtikz{Заблокированный прыжок по диагонали}{jps_blocked_diagonal_jump}
{
	\mybox[x=1,y=1,color=grey]
	\mybox[x=1,y=2,color=grey]
	\mybox[x=1,y=3,color=grey]
	\mybox[x=2,y=1,color=grey]
	\mybox[x=3,y=1,color=grey]
	\mybox[x=3,y=1,color=grey]
	\mybox[x=2,y=3,color=black]
	\mybox[x=3,y=2,color=black]
	\mybox[x=2,y=2,color=green]
	
	\mygrid[width=5, height=5]
	
	\myarrow[startx=0,starty=0,endx=1,endy=1,color=darkgrey]
	\myarrow[startx=1,starty=1,endx=1,endy=2,color=darkgrey]
	\myarrow[startx=1,starty=2,endx=1,endy=3,color=darkgrey]
	\myarrow[startx=1,starty=1,endx=2,endy=1,color=darkgrey]
	\myarrow[startx=2,starty=1,endx=3,endy=1,color=darkgrey]
	\myarrow[startx=1,starty=1,endx=2,endy=2,color=darkgrey]
	\myarrow[startx=2,starty=2,endx=2,endy=3,color=red]
	\myarrow[startx=2,starty=2,endx=3,endy=2,color=red]
	\myarrow[startx=2,starty=2,endx=3,endy=3,color=red]
}
  
Горизонтальный и вертикальный прыжки происходят пока следующая клетка проходима, не является конечной клеткой и находится на карте. На каждом шаге происходит проверка на наличие вынужденных соседей и если такой сосед найден, то он добавляется в открытый список и метод прерывается. 

Когда открытый список исчерпан - происходит восстановление пути так же как и в A*.

\subsection{Реализация и интеграция GoalBounds}

Алгоритм Goal Bounding можно разделить на два этапа: этап препроцессинга и проверка направления во время выполнения поиска пути. Во время препроцессинга происходит прохождение по всем клеткам карты, для каждой клетки выполняет волновой алгоритм, который заполняет все клетки стоимостью путей до них от начальной клетки и изначальным направлением по которому алгоритм пришёл в данную клетку. После чего для данной клетки определяется 8 ограничивающих прямоугольников \cref{fig:goal_bounding_dbg}, по одному на каждое направление \cref{fig:goal_bounding_fill}. Определяются они по такому алгоритму: 

\begin{enumerate}
	\item делаем минимальную точку прямоугольника равную координатам правой верхней точки карты, а максимальную - равной левой нижней;
	\item идём по всем точкам и расширяем прямоугольник соответствующий направлению в котором точка была достигнута из начальной, что бы прямоугольник включал её. 
\end{enumerate}

\begin{figure}[!htb]
	\centering
	\captionsetup{justification=centering}
	\begin{lstlisting}
auto& bbNode = m_bbMap.GetNodeBB( startX, startY );
for( uint16_t i = 0; i < map.GetSizeX(); i++ )
{
	for( uint16_t j = 0; j < map.GetSizeY(); j++ )
	{
		 auto& ffNode = ffMap->GetNode( i, j ); 
		 if( map.IsPassable( i, j, type ))
		 {
			 auto& dirBB = bbNode.boundingBoxes[static_cast<int>(ffNode.originalDir)];	 
			 dirBB.minX = std::min( dirBB.minX, i );
			 dirBB.maxX = std::max( dirBB.maxX, i );
			 dirBB.minY = std::min( dirBB.minY, j );
			 dirBB.maxY = std::max( dirBB.maxY, j );
		 }
	 }
 }
	\end{lstlisting}
	\caption{Определение ограничивающих прямоугольников}
	\label{fig:goal_bounding_fill}
\end{figure}

\addimg{img/goal_bounding_dbg.png}{0.8}{Часть карты с раскрашенными в разные цвета ограничивающими прямоугольниками}{goal_bounding_dbg}

Так как алгоритм вычисляет ограничивающие прямоугольники для каждой клетки независимо, то его можно распараллелить. Для этого был использован пул поток, реализованный библиотекой threadpool11. Для каждой клетки создаётся своя задача и добавляется в очередь, после добавления всех задач происходит ожидание их завершения. Каждый поток имеет свою карту для волнового алгоритма которая создаётся один раз для одного потока. После исполнения алгоритма результат записывается в файл рядом с файлом карты, в файле поочерёдно, для каждой клетки, записано 32 целочисленных значения - четыре значения для каждого из восьми направлений из клетки, которые являются верхней левой и правой нижней точкой прямоугольника. Если файл с результатами вычислений уже существует на диске, то загружается он, что бы не проводить трудоёмкие вычисления заново.

Для включения Goal Bounding в A* требуется добавить проверку на правильность направления сразу после проверки на вхождение точки в границу карты при циклическом обходе соседей текущей точки \cref{fig:a_star_test_goal_bounds}. 

Для добавления Goal Bouding в JPS требуется провести аналогичное действие, но для всех функций прыжков.

\begin{figure}[!htb]
	\centering
	\captionsetup{justification=centering}
	\begin{lstlisting}
for( int i = 0; i < 8; i++ )
{
	const auto& delta = coordDeltas[i];
	const CoordsType nextX = best.x + delta.first;
	const CoordsType nextY = best.y + delta.second;
	if( m_map.IsNodeOnMap( nextX, nextY ))
	{
		if( TestGoalBounding( best.x, best.y, targetX, targetY, dirs[i] ))
		{
			const MapNode& node = m_map.GetNode( nextX, nextY );
			if( node.IsPassable( unitType ))
			{
	\end{lstlisting}
	\caption{Проверка Goal Bounds}
	\label{fig:a_star_test_goal_bounds}
\end{figure}

\subsection{Реализация визуализатора}

Для реализации визуализатора были использованы кроссплатформенные библиотеки SFML и SFGUI. Карта рисуется несколькими слоями, которые для удобства и скорости отрисовываются в отдельные текстуры. Визуализатор рисует такие слои: слой карты, последний найденный путь, открытые клетки алгоритмом A*, направления прыжков алгоритма JPS, ограничивающие прямоугольники алгоритма Goal Bounding для выбранной клетки.

Нахождение пути происходит после выбора начальной и конечной точки, после чего он отображается на карте и выводится приблизительное время выполнения алгоритма. 

Визуализатор, кроме отображения путей с отладочной информацией имеет возможность сравнивать скорость выполнения алгоритмов для выбранного пути.